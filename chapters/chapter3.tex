\documentclass[../main.tex]{subfiles}

\begin{document}

\section{Exercise 3.1}

Code:

\begin{lstlisting}
(define (make-accumulator sum)
  (lambda (x)
    (begin (set! sum (+ sum x))
           sum))))
\end{lstlisting}

Tests:

\begin{lstlisting}
(define acc-5 (make-accumulator 5))
(acc-5 1) ; 6 expected
(acc-5 2) ; 8 expected
(acc-5 3) ; 11 expected
(acc-5 4) ; 15 expected
(acc-5 -15) ; 0 expected
(define acc-neg-1 (make-accumulator -1))
(acc-neg-1 11) ; 10 expected
\end{lstlisting}

Result:

\begin{lstlisting}
6  ; pass
8  ; pass
11 ; pass
15 ; pass
0  ; pass
10 ; pass
\end{lstlisting}

\section{Exercise 3.2}

Code:

\begin{lstlisting}
(define (make-monitored proc)
  (let ((count 0))
    (define (dispatch . args)
      (cond ((equal? args '(how-many-calls?)) count)
            ((equal? args '(reset-count))
             (begin (set! count 0)
                    'ok))
            (else (begin (set! count (+ 1 count))
                         (apply proc args)))))
    dispatch))
\end{lstlisting}

Tests:

\begin{lstlisting}
(define s (make-monitored sqrt))
(s 'how-many-calls?) ; 0 expected
(s 100) ; 10 expected
(s 49) ; 7 expected
(s 1) ; 1 expected
(s 0) ; 0 expected
(s 'how-many-calls?) ; 4 expected
(s 'reset-count) ; ok expected
(s 'how-many-calls?) ; 0 expected
(s 144) ; 12 expected
(s 64) ; 8 expected
(s 4) ; 2 expected
(s 0) ; 0 expected
(s 'how-many-calls?) ; 4 expected
\end{lstlisting}

Result:

\begin{lstlisting}
;; procedure s defined
0  ; pass
10 ; pass
7  ; pass
1  ; pass
0  ; pass
4  ; pass
ok ; pass
0  ; pass
12 ; pass
8  ; pass
2  ; pass
0  ; pass
4  ; pass
\end{lstlisting}

\section{Exercise 3.3}

Code:

\begin{lstlisting}
(define (make-account balance password)
  (define (withdraw amount)
    (if (>= balance amount)
        (begin (set! balance (- balance amount))
               balance)
        "Insufficient funds"))
  (define (deposit amount)
    (set! balance (+ balance amount))
    balance)
  (define (dispatch pwd m)
    (cond ((not (eq? pwd password))
           (lambda (amount) "Incorrect password"))
          ((eq? m 'withdraw) withdraw)
          ((eq? m 'deposit) deposit)
          (else "Unknown method")))
  dispatch)
\end{lstlisting}

Tests:

\begin{lstlisting}
(define my-account (make-account 100 'take-that))
((my-account 'take-that 'withdraw) 50) ; 50 expected
((my-account 'take-that 'deposit) 30) ; 80 expected
((my-account 'take-this 'withdraw) 80) ; error expected
((my-account 'take-that 'withdraw) 80) ; 0 expected
\end{lstlisting}

Results:

\begin{lstlisting}
;; my-account defined
50 ; pass
80 ; pass
"Incorrect password" ; pass
0 ; pass
\end{lstlisting}

\section{Exercise 3.4}

Code:

\begin{lstlisting}
(define (call-the-cops)
  "Hey you b**stard!")
(define (make-account balance password)
  (let ((error-count 0))
    (define (withdraw amount)
      (if (>= balance amount)
          (begin (set! balance (- balance amount))
                 balance)
          "Insufficient funds"))
    (define (deposit amount)
      (set! balance (+ balance amount))
      balance)
    (define (error amount)
      (if (< error-count 7)
          (begin (set! error-count (+ error-count 1))
                 "Incorrect password")
          (call-the-cops)))
    (define (dispatch pwd m)
      (if (not (eq? pwd password))
          error
          (begin (set! error-count 0)
                 (cond ((eq? m 'withdraw) withdraw)
                       ((eq? m 'deposit) deposit)
                       (else "Unknown method")))))
    dispatch))
\end{lstlisting}

Tests:

\begin{lstlisting}
(define my-account (make-account 100 'take-that))
((my-account 'take-this 'withdraw) 50) ; error expected
((my-account 'take-this 'withdraw) 50) ; error expected
((my-account 'take-this 'withdraw) 50) ; error expected
((my-account 'take-this 'withdraw) 50) ; error expected
((my-account 'take-this 'withdraw) 50) ; error expected
((my-account 'take-this 'withdraw) 50) ; error expected
((my-account 'take-this 'withdraw) 50) ; error expected
((my-account 'take-this 'withdraw) 50) ; cops expected
((my-account 'take-that 'withdraw) 50) ; 50 expected
\end{lstlisting}

Result:

\begin{lstlisting}
"Incorrect password" ; pass
"Incorrect password" ; pass
"Incorrect password" ; pass
"Incorrect password" ; pass
"Incorrect password" ; pass
"Incorrect password" ; pass
"Incorrect password" ; pass
"Hey you b**stard!"  ; pass
50                   ; pass
\end{lstlisting}

\section{Exercise 3.5}

Code:

\begin{lstlisting}
(define (monte-carlo trials experiment)
  (define (iter trials-remaining trials-passed)
    (cond ((= trials-remaining 0)
           (/ (+ 0.0 trials-passed) trials))
          ((experiment)
           (iter (- trials-remaining 1) (+ trials-passed 1)))
          (else
           (iter (- trials-remaining 1) trials-passed))))
  (iter trials 0))
(define (estimate-integral P xl xu yl yu trials)
  (define (rand-range lower upper)
    (+ lower (random (- upper lower))))
  (define (experiment)
    (let ((x (rand-range xl xu))
          (y (rand-range yl yu)))
      (P x y)))
  (monte-carlo trials experiment))
\end{lstlisting}

Tests (same test for 5 times):

\begin{lstlisting}
(* 4
   (estimate-integral
     (lambda (x y)
       (< (+ (* x x) (* y y)) 1))
     -1.0 1.0 -1.0 1.0 100000)) ;; 3.14159 expected
\end{lstlisting}

Result:

\begin{lstlisting}
3.14116 ; pass
3.15324 ; pass
3.13736 ; pass
3.132   ; pass
3.14752 ; pass
\end{lstlisting}

\section{Exercise 3.6}

To test the code, a pseudo \lstinline{rand-update}
 procedure and a constant initial seed are used.

Code:

\begin{lstlisting}
(define rand
  (let ((x 2333))
    (define (rand-update x)
      (+ x 1))
    (define (generate)
      (set! x (rand-update x))
      x)
    (define (reset value)
      (set! x value)
      'ok)
    (define (dispatch method)
      (cond ((eq? method 'generate) (generate))
            ((eq? method 'reset) reset)
            (else "Unknown method")))
    dispatch))
\end{lstlisting}

Tests:

\begin{lstlisting}
((rand 'reset) 0) ; ok expected
(rand 'generate)  ; 1 expected
(rand 'generate)  ; 2 expected
(rand 'generate)  ; 3 expected
((rand 'reset) 0) ; ok expected
(rand 'generate)  ; 1 expected
(rand 'generate)  ; 2 expected
(rand 'generate)  ; 3 expected
\end{lstlisting}

Result:

\begin{lstlisting}
ok ; pass
1  ; pass
2  ; pass
3  ; pass
ok ; pass
1  ; pass
2  ; pass
3  ; pass
\end{lstlisting}

\end{document}










